\chapter{Serverless architecture for web applications}

% różne podejścia do budowy aplikacji webowych - jak to przekłada się na architekturę serverless?
% czy można wyznaczyć ogólne wytyczne pozwalające na efektywne tworzenie aplikacji webowych w architekturze serverless?
% ---

reference to previous chapter, how serverless relates to:
- 1. distributed system - monolith -> microservices -> FaaS as next step
- 2. database - what are the requirements? consistency vs. availability
- 3. clients - thin -> thick (-> thin?) -> CDN / Lambda@Edge to render static pages, 3.5. client/server communication - REST / Websocket / GraphQL - GraphQL using AppSync

Serverless should be sufficient for what's needed to develop modern web application 

1. Server

- Serverless - event-driven architecture, require building application compliant with that approach

---

- What if processing is not compliant? - hybrid solution?
- what with limitations? - interactive clients, low latency / "real-time" 

2. Database

- "Classic" databases are not suitable for serverless architecture - needs redefinition - DynamoDB, Aurora - investigation how these works? how does these affects the application development?
- functions are stateless - 3rd party components to persist state

3. Clients 

- does not change much or does it?
- how serverless (cdn, static site rendering using lambda@edge) can be used to make it better?
- sdk/libraries for building clients

---
% How to leverage serverless computing for developing web applications?
% Types of processing not suitable for serverless computing?
% https://aws.amazon.com/lambda/resources/reference-architectures/
% https://docs.aws.amazon.com/whitepapers/latest/modern-application-development-on-aws/modern-application-development-on-aws.pdf
% https://www.thoughtworks.com/insights/blog/traits-serverless-architecture