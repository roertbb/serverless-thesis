\chapter{Serverless architecture for web applications}

% różne podejścia do budowy aplikacji webowych - jak to przekłada się na architekturę serverless?
% czy można wyznaczyć ogólne wytyczne pozwalające na efektywne tworzenie aplikacji webowych w architekturze serverless?
% ---

% reference to previous chapter, how serverless relates to:
% - 1. distributed system - monolith -> microservices -> FaaS as next step
% - 2. database - what are the requirements? consistency vs. availability
% - 3. clients - thin -> thick (-> thin?) -> CDN / Lambda@Edge to render static pages, 3.5. client/server communication - REST / Websocket / GraphQL - GraphQL using AppSync

% Serverless should be sufficient for what's needed to develop modern web application 

% 1. Server

% - Serverless - event-driven architecture, require building application compliant with that approach

% ---

% - What if processing is not compliant? - hybrid solution?
% - what with limitations? - interactive clients, low latency / "real-time" 

% 2. Database

% - "Classic" databases are not suitable for serverless architecture - needs redefinition - DynamoDB, Aurora - investigation how these works? how does these affects the application development?
% - functions are stateless - 3rd party components to persist state

% 3. Clients 

% - does not change much or does it?
% - how serverless (cdn, static site rendering using lambda@edge) can be used to make it better?
% - sdk/libraries for building clients

% ---

% % How to leverage serverless computing for developing web applications?
% % Types of processing not suitable for serverless computing?
% % https://aws.amazon.com/lambda/resources/reference-architectures/
% % https://docs.aws.amazon.com/whitepapers/latest/modern-application-development-on-aws/modern-application-development-on-aws.pdf
% % https://www.thoughtworks.com/insights/blog/traits-serverless-architecture

% ---

% - A Review of Serverless Use Cases and their Characteristics
% - LEVERAGING SERVERLESS CLOUD COMPUTING ARCHITECTURES
% - Patterns for Serverless Functions (Function-as-a-Service): A Multivocal Literature Review

% % Migrating a web application to serverless architecture - serverless design patterns in more details
% % https://jyx.jyu.fi/bitstream/handle/123456789/64836/URN%3ANBN%3Afi%3Ajyu-201906253422.pdf?sequence=1

% ---

% Research Questions:

% 1. Is Faas appropriate approach for building web applications - lower cons, better reliability, better performance
% - FaaS serverless as Backend ~ Microservices

% \cite{BerkeleyServerless}
% 3 Limitations of Today’s Serverless Computing Platforms

% - can serverless handle some finer grained communication - comparing to the serverfull solution? - drawing / excalidraw like

% \cite{LeveragingServerlessCloudComputingArchitectures}
% - These serverless computing services enable creating microservices, without the need to manage servers, that can easily be deployed and are automatically scaled

% ---

% 2. Database for serverless - communicating with database

% \cite{BerkeleyServerless}
% Serverless SQLite: Databases

% ---

% 3. Client

% - websockets / real-time communication / pushing update to the client - push to pull pattern - as being sustainable to be implemented utilising lambdas

% ---

% 4. Design patterns / good practice
% - basic approach / FaaS -> optimised FaaS

% ---

% % Mikhail Shilkov - interesting blog around serverless, AWS, Azure, Pulumi
% % - https://mikhail.io/serverless/coldstarts/big3/

% ---

\section{Intro}

% TODO:rb :eyes:
% https://aws.amazon.com/blogs/compute/building-well-architected-serverless-applications-controlling-serverless-api-access-part-1/
% https://www.serverless.com/blog/how-create-rest-api-serverless-components
% https://www.andmore.dev/blog/build-serverless-api-with-no-lambda/
% https://www.mdpi.com/2073-431X/8/2/50/htm

Describe how serverless relate to microservices?

\begin{enumerate}
    \item multiple components combined to perform application logic
    \item event driven - triggers other components, lambda for transforming the data
    \item scaling independent components
    \item ...
\end{enumerate}

Research questions:

\begin{enumerate}
    \item Is Faas appropriate approach for building web applications (FaaS serverless as Backend similar Microservices)? - lower cont, better reliability, better performance
    \item Server tier?
    \item Database for serverless - communicating with database
    \item Client
    \item Design patterns and good practice (basic approach / FaaS -> optimised FaaS)
\end{enumerate}

why AWS? how the research looks like?

\section{Is Faas appropriate approach for building web applications?}

how serverless fulfills the requirements put in front of the web applications?

---

better performance

\begin{itemize}
    \item cold starts - pre-warming functions, provisioned concurrency
    \item serverless usually should not be faster than serverful
    \item utilising parallel execution for faster processing of the tasks (regarding the latex parsing - splitting into chunks that can be parsed together)
\end{itemize}

better reliability

\begin{itemize}
    \item implicit failover
    \item DLQ in SQS and others?
\end{itemize}

and so on

---

lower cost

\begin{itemize}
    \item some types of processing are not sufficient - \cite{BerkeleyServerless}
    \item some types of processing can be redesigned to leverage benefits of serverless - \cite{BerkeleyServerless}
    \item calculating betteruptime.com on FaaS - more expensive than serverful on AWS, which is more expensive than VPS somewhere else
    \item conditions/characteristics/decission tree - when serverless will be more expensive
\end{itemize}

---

examples

\begin{itemize}
    \item small web app - processing receipts/invoices
    \item event driven processing - parsing latex files to interactive presentations
\end{itemize}

\section{Server Tier}

\begin{itemize}
    \item serverless processing characteristics - event driven - lambdas for transforming data
    \item distributed sagas - step functions
    \item good practices - using SQS for batching, streams from DynamoDB, using the lambda only to transform the data
\end{itemize}

\section{Database Tier}

\begin{itemize}
    \item relational database not suitable for serverless - RDS proxy holding connections
    \item DynamoDB, Aurora - how they work?
    \item DynamoDB modeling? single table design - cost savings, increase performance
\end{itemize}

\section{Clients}

\begin{itemize}
    \item APIGateway - REST + websockets
    \item AppSync - GraphQL
    \item serverless loves async processing - push to pull pattern - pushing update to client
    \item hosting static website from S3 + CloudFront - hosting client
\end{itemize}

\section{Desing patterns and good practices}