\chapter{Serverless computing}

\section*{Origins}

- next generation of services utilizing cloud computing in pay as you go model
- IaaS - outsourcing infrastructure management, PaaS - outsourcing application management, ...
- 2014 AWS Lambda

\section*{Definition of serverless computing}

Already used and adopted by cloud providers since a few years, but there's no concise and clear definition of what serverless is

characteristics:

- do not require infrastructure management
- scaled, billed based on usage
- deployment granularity - starting from single function

2 components:

- FaaS - single, stateless function running in ephemeral container, triggered by some event, leveraging scaling of demand
- BaaS - 3rd party service used as building blocks for most frequently incorporated features in apps development - database, authentication and authorization, queue and pub/sub mechanisms or even hosting, notifications, deployments and monitoring

Building apps in serverless architecture is combination of the 2 components mentioned above - programmers write "glue code"

\section*{BaaS}

- web app and mobile app domination
- similar set of features - need to store data, files, authentication, authorization
- cloud providers growth

Based on that various services emerged. When properly integrated, they provide common functionalities need when building web application + brings the benefits of cloud processing - autoscaling, cost efficiency etc.

Services provided by vendors are pretty similar to their standard equivalents, but we're outsourcing processing, storage, responsibility for management to provider.

\subsection*{BaaS in web applications}

- database
- storage
- authentication and authorization
- etc.

\section*{FaaS}

- responsible for processing - building block of business logic
- new appriach in terms of processing and code structure

FaaS:

- set of small, stateless functions
- restricted execution duration
- invoked in response to an event - http gateway, db operation, message from queue
- running in ephemeral container
- executed in parallel - provider handles scaling

\subsection*{Execution and lifecycle}

- warm start and cold start
- scheduling
- how the process of executing lambda works
- stateless - containers are ephemeral, we can't rely on the data preserved inside function container, usage of external services for persistance - BaaS, that makes functions efectivaly stateless
- container - created upon start of execution, removed when execution finishes, share no state between execution - can be assumed that 1 container = 1 function execution, but container can be reused. Automatically scaled, provisioning contaienrs - vital for scaling, instantioation = startup latency, managind resources in multi-tenant environment - resources can be exhausted, how to preserve security - more in benefits and challenges
- event driven function invocation. Guarantees - can be invoked 1 or more, can invoke 1 or more functions, versioning, many event sources. Types of events - Endpoint (HHTP Gateway), messaging event (queue, pub/sub), storage, scheduled (cron), etc.

\section*{Benefits and challenges}

\subsection*{Benefits}

- reduced operational cost
- reduced scaling cost and automating autoscaling
- reduced time to market
- development opportunities
- greener computing

\subsection*{Challenges}

- vendor dependence - downtime on provide's side
- security
- vendor lock-in
- multi tenancy problems
- testing
- performance
- complexity of development
- complexity od deployment

\section*{Characteristics of serverless architecture}

features:

- no resource management
- auto-scaling
- pay per usage
- event driven architecture

\section*{[?] Implication of serverless architecture on building web apps}