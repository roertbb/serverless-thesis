\chapter{Introduction}

\section{Motivation}

Serverless computing is a new approach to developing and managing applications and services, which is gaining popularity every year.
It changes the perspective of the application architecture, which is based on the stateless and short-lived serverless functions, executed on demand to process the workloads.
Serverless computing heavily relies on~the~integration of external services hosted in the cloud, incorporating them in the developed solution to provide the core application logic and additional capabilities.
Both of~the~mentioned components are executed in the cloud environment, without the need to~manage the~infrastructure and with a pricing model proportional to the used resources.
It makes the~serverless paradigm an appealing solution to numerous companies, thanks to the benefits it can bring by reducing the cost and improving the development agility, focusing only on~the~business logic.

The field is still evolving, with industry-leading cloud vendors introducing new services and extending existing ones with new capabilities.
However, serverless computing is still not a mature solution, lacking proper standardization.
Building efficient solutions using the serverless architecture is not a trivial task, which requires broad domain knowledge and practical experience when working with the services hosted in the cloud platforms.
The~new~approach changes the way how applications are developed, deployed and executed.

At the same time, the field of web applications expanded significantly, making them an~appealing medium for various companies to modernise their businesses or startups to~introduce new products.
The widespread availability of the Internet, as well~as~the~computers and mobile devices, contributed to the significant grow of available services, covering many areas of human life.
The development does not only concern the number of available web applications, but also the importance and complexity of performed operations and the volumes of processed data.
The provided solutions need to work efficiently, scale according to the number of active users, while at the same time ensure security and compliance, along with a rich and intuitive interface, similar to native applications.

\section{Objective and scope of research}

The key objective of the thesis is to conduct research on how the serverless processing and Function as a Service model can be applied in the web application field.
To provide the comprehensive analysis several objectives need to be fulfilled.

In the beginning, it is crucial to provide the necessary context by introducing the field of serverless computing and describing the Function as a Service model.
The thorough analysis of benefits and challenges of the researched domain should be identified, along with the comparison of the offerings from numerous cloud providers as well as the overview of existing use cases of serverless architecture.

Moreover, the field of web applications needs to be introduced, summarising the requirements they need to fulfill, outlining the architectures of web applications developed nowadays and discussing their emerging problems.

Thorough introduction of both fields should be sufficient to highlight the crucial aspects in the further analysis.
The research of the serverless architecture application in the field of the web application should be conducted, covering the topics of applicability of the discussed architecture,
highlighting the limitations and possible solutions and outlining the recommended architectures.
In addition to that, providing hands-on implementations along with their further analysis should illustrate the possibilities of the serverless architecture as well as provide a deeper understanding and practical experience of the researched domain.

Finally, the results of the analysis should be concluded, highlighting the key capabilities and possibilities of the serverless architecture in the web application field.

\section{Structure of thesis}

The thesis is divided into five chapters. Each of them introduces an important step in~the research process, covering the domain knowledge as well as providing more in depth analysis of the researched area.

The second chapter introduces the serverless processing field.
In the chapter, origins and notion of serverless computing is defined.
Next, the serverless components are analysed in more detail, providing the information about the Function as a Service model.
The~overview of benefits and challenges is provided to gain a deeper understanding of~the~researched architecture.
Finally, the comparison of available cloud vendors and offerings of their serverless platform is discussed, along with the example use cases, extending the~context by real-world examples.

The third chapter is devoted to the web application domain. The origins and further evolution of web application is described.
Furthermore, the requirements which are put in~front of the web applications nowadays are thoroughly discussed, providing a direction for~further analysis.
At the end of the chapter, the architectures of modern web applications are covered in more detail, highlighting the solutions used currently and discussing the~trade-offs they create.

The fourth chapter is the main contribution of the thesis, providing comprehensive analysis of the serverless processing applicability in the web application field.
Firstly,~the~research questions are formed and the research approach is specified.
Furthermore, the~topic of the serverless suitability for the web applications is thoroughly analysed.
The~practical implementations of the web applications using the serverless architecture are~provided to~give more hands-on insight into the researched field.
Next, detailed overview of~the~FaaS and serverless processing model is conducted, including an extensive analysis of~one~of~the~practical implementations, in the form of a case study.
Moreover, the area of suitability of the datastores in the serverless architecture and the serverless databases is presented.
Lastly, the discussion of the web application client architecture and used communication patterns are covered.

The fifth and final chapter is a summary of the research, referring back to the formed research questions and highlighting the potential directions of further research.

